
\chapter{Another Chapter}

\kant[2]

Some test citation~\cite{bach_time_2016}. \kant[1]

\begin{equation}
    1 + 2 +3 +4 + ... + n = \frac{n^2 + n}{2}
    \label{eq:gauss}
\end{equation}

\begin{figure}
    \centering
    \includegraphics[height=4cm]{example-image-a}
    \caption{An example figure. It has a long caption. This way, the formatting of captions that span multiple lines becomes visible.}
\end{figure}

\kant[1-10]