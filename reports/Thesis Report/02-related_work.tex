
\chapter{Related Work}

a lot of text with cites\cite{python_maps}

cite of "Bergauf"\cite{bergauf_1_2025}

1: geopolitical data overview
2: time series Aigner https://link.springer.com/book/10.1007/978-1-4471-7527-8
3: related work


new section: data and task description as an overview (from the introduction)



\begin{wrapfigure}{r}{0.48\textwidth}
  \centering
  \includegraphics[width=0.46\textwidth]{reports/Thesis Report/figures/german_temperature.png}
  \caption{Average temperature in Germany, October 2018.
  Source: Copernicus Climate Change Service (C3S) \cite{earthdatahub_tutorials}.}
  \label{fig:german_temperature}
\end{wrapfigure}

The ERA5 dataset is widely used in climate and cryosphere research.
Numerous glacier-related studies are based on ERA5-derived products, for example the work of Fabien Maussion and collaborators \parencite{maussion_publications}.
However, these studies typically focus on static analyses, and interactive exploration of the underlying data is rarely supported.

Copernicus Climate Change Service provides a collection of Jupyter Notebook tutorials that serve as an entry point for exploring ERA5 data \parencite{earthdatahub_tutorials}.
While these notebooks demonstrate typical workflows, user interaction is limited and usually requires direct modification or extension of the underlying code.


\begin{wrapfigure}{l}{0.55\textwidth}
  \centering
  \includegraphics[width=0.53\textwidth]{reports/Thesis Report/figures/surface_temp_berlin.png}
  \caption{Temperature development over time in Berlin \cite{earthdatahub_tutorials}.}
  \label{fig:surface_temp_berlin}
\end{wrapfigure}



