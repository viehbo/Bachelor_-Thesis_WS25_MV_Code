\chapter{Related Work}

The ERA5 dataset is widely used in climate and cryosphere research.
Numerous glacier-related studies are based on ERA5-derived products, for example the work of Fabien Maussion and collaborators \parencite{maussion_publications}.

\begin{wrapfigure}{r}{0.65\textwidth}
  \centering
  \includegraphics[width=0.55\textwidth]{reports/Thesis Report/figures/era5_ts_global_glaciers.png}
  \caption{Temperature change at glacier locations (area-weighted) compared to the global mean \cite{maussion_publications}.}
  \label{fig:era5_ts_global_glacier}
\end{wrapfigure}

As shown in Figure~\ref{fig:era5_ts_global_glacier}, the temperature change at glacier locations is stronger than the global average.
However, these studies typically focus on static analyses, and interactive exploration of the data is usually not possible.

\par\medskip

Copernicus Climate Change Service provides as a starting point several Jupyter Notebooks, where ERA5 data can be discovered \parencite{earthdatahub_tutorials}.
The interaction for a user is very restricted and usually requires changing or adding code.

\begin{wrapfigure}{r}{0.45\textwidth}
  \centering
  \includegraphics[width=0.55\textwidth]{reports/Thesis Report/figures/german_temperature.png}
  \caption{Average temperature in Germany, October 2018 \cite{earthdatahub_tutorials}.}
  \label{fig:german_temperature}
\end{wrapfigure}

\vspace{1em}

An example visualization is shown in Figure~\ref{fig:german_temperature}.
This is grat to make the first climps on it, but not enough to decide, is the data are suitable for a potetial ML task.

\vspace{1em}

WGMS provides with the \textit{Fluctuations of Glaciers Browser} \cite{arcgis_experience_836c66d} a browser tool to discover glacier on a map.
Over 100,000 glaciers are coverd.
One can get insights about the database from this glacier measurements.
But whats not possible, is to interact with measurements or directly compare them.
What is missing is an interactive tool to gain insights into ERA5 weather data and WGMS glacier data.


\vspace{1em}


\begin{figure}[htbp]
  \centering
  \includegraphics[width=0.70\textwidth]{reports/Thesis Report/figures/dachstein_glacier.png}
  \caption{Dachstein glaciers visualized via the \textit{Fluctuations of Glaciers Browser} \cite{arcgis_experience_836c66d}.}
  \label{fig:dachstein_glacier}
\end{figure}

\vspace{1em}


\begin{figure}[htbp]
  \centering
  \includegraphics[width=0.2\textwidth]{reports/Thesis Report/figures/surface_temp_berlin.png}
  \caption{Temperature development over time in Berlin \cite{earthdatahub_tutorials}.}
  \label{fig:surface_temp_berlin}
\end{figure}