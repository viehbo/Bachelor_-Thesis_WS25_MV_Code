\chapter{Visualization}

The objective is to create a interactive tool as a web-interface, with that a person can intuitively gain insights to the data. The project from the Seminar in AI already gives some learning's, which are the stating point of this work.

\vspace{1em}
\textbf{The concept consists of the following parts:}

\begin{itemize}
    \item Set conceptual design choices
    \item Collecting ERA5 and glacier data
    \item Data preprocessing
    \item Setup the framework of the visualization
    \item Implement one element / function after another
\end{itemize}


\section{Conceptual design choices}

Goal is to find the optimal tradeoff between a maximum of interactivity and a intuative interface. For this, the user interface is seperated into 4 sections.

\begin{figure}[htbp]
\centering
\begin{tikzpicture}[
  box/.style={draw, line width=0.6pt, align=center, minimum height=1.2cm},
  smallbox/.style={box, minimum width=5.8cm},
  tallbox/.style={box, minimum width=4.2cm, minimum height=6.2cm},
  widebox/.style={box, minimum width=10.9cm},
  font=\small
]

% Top: file selection
\node[widebox] (top) {File selection area};

% Left: controls
\node[tallbox, below=0.9cm of top.west, anchor=north west] (ctrl) {Adjustments\\and controls};

% Right: map + line plot
\node[smallbox, right=0.9cm of ctrl.north east, anchor=north west, minimum height=3.0cm] (map) {Map};
\node[smallbox, below=0.8cm of map, minimum height=2.4cm] (plot) {Line plot};

\end{tikzpicture}
\caption{Conceptual layout of the user interface, showing the separation of data selection, parameter controls, and visualization views.}
\label{fig:ui_layout}
\end{figure}

The workflow is starting from the top-left and ends on the bottom-right. Like when reading a book. This should provide the basic structure for an intuitive handling.
Of course, in all sections, interaction will be possible. But they will influence each other only top-down:
\textbf{File selection} $\rightarrow$ \textbf{Adjustments and controls} $\rightarrow$ \textbf{Map} $\rightarrow$ \textbf{Line Plot}


\section{Data Preprocessing}

ECMWF provides ERA5 data in GRIB\cite{ecmwf_grib_explanation} format. The GRIB format is designed to storing data and distribute them. There are Python libraries, which can read this format but for the visualisation task in this project, GRIB-files are not ideal.

According to \textcite[pp.~85--87]{python_maps}, the library \textit{Xarray} is optimized for performing labeled multi-dimensional computations and subsequent visualization of geospatial data. 
Working directly with GRIB files, however, can create a computational bottleneck in this environment due to their structure and decoding requirements. 
Both \textit{Python Maps} \parencite[p.~88]{python_maps} and the official Xarray documentation \parencite{xarray_documentation} recommend the use of the NetCDF format for efficient storage and processing of scientific data. 
NetCDF provides a self-describing and platform-independent data model that is particularly well suited for climate and meteorological datasets \parencite{netcdf_documentation}.



\section{User Interface}

Parts and interaction, design choice 


\section{Implementation}

Panel + Bokeh



