\chapter{Visualization}

The objective is to create a interactive tool as a web-interface, with that a person can intuitively gain insights to the data. The project from the Seminar in AI already gives some learning's, which are the stating point of this work.

\vspace{1em}
\textbf{The concept consists of the following parts:}

\begin{itemize}
    \item Set conceptual design choices
    \item Collecting ERA5 and glacier data
    \item Data preprocessing
    \item Setup the framework of the visualization
    \item Implement one element / function after another
\end{itemize}


\section{Conceptual design choices}

Goal is to find the optimal tradeoff between a maximum of interactivity and a intuative interface. For this, the user interface is seperated into 4 sections.

\begin{figure}[htbp]
\centering
\begin{tikzpicture}[
  box/.style={draw, line width=0.6pt, align=center, minimum height=1.2cm},
  smallbox/.style={box, minimum width=5.8cm},
  tallbox/.style={box, minimum width=4.2cm, minimum height=6.2cm},
  widebox/.style={box, minimum width=10.6cm},
  font=\small
]

% Top: file selection
\node[widebox] (top) {File selection area};

% Left: controls
\node[tallbox, below=0.9cm of top.west, anchor=north west] (ctrl) {Adjustments\\and controls};

% Right: map + line plot
\node[smallbox, right=0.9cm of ctrl.north east, anchor=north west, minimum height=3.0cm] (map) {Map};
\node[smallbox, below=0.8cm of map, minimum height=2.4cm] (plot) {Line plot};

\end{tikzpicture}
\caption{Conceptual layout of the user interface, showing the separation of data selection, parameter controls, and visualization views.}
\label{fig:ui_layout}
\end{figure}




\section{Data Preprocessing}

from ERA5 .grib to netCDF4



\section{User Interface}

Parts and interaction, design choice 


\section{Implementation}

Panel + Bokeh



